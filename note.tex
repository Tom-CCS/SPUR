\documentclass{article}
\usepackage[utf8]{inputenc}


\usepackage{listings}
\usepackage[
nochapters, % Turn off chapters since this is an article        
beramono, % Use the Bera Mono font for monospaced text (\texttt)
pdfspacing, % Makes use of pdftex’ letter spacing capabilities via the microtype package
dottedtoc % Dotted lines leading to the page numbers in the table of contents
]{classicthesis} % The layout is based on the Classic Thesis style

\usepackage{arsclassica} % Modifies the Classic Thesis package

\usepackage[T1]{fontenc} % Use 8-bit encoding that has 256 glyphs

\usepackage[utf8]{inputenc} % Required for including letters with accents

\usepackage{graphicx} % Required for including images
\graphicspath{{Figures/}} % Set the default folder for images

\usepackage{enumitem} % Required for manipulating the whitespace between and within lists

\usepackage{lipsum} % Used for inserting dummy 'Lorem ipsum' text into the template

\usepackage{subfig} % Required for creating figures with multiple parts (subfigures)

\usepackage{amsmath,amssymb,amsthm} % For including math equations, theorems, symbols, etc

\usepackage{varioref} % More descriptive referencing
\usepackage[mathscr]{euscript}
\let\euscr\mathscr \let\mathscr\relax% just so we can load this and rsfs
\usepackage[scr]{rsfso}
\newcommand{\powerset}{\raisebox{.15\baselineskip}{\Large\ensuremath{\wp}}}
%----------------------------------------------------------------------------------------
%	THEOREM STYLES
%---------------------------------------------------------------------------------------

\theoremstyle{definition} % Define theorem styles here based on the definition style (used for definitions and examples)
\newtheorem{definition}{Definition}

\theoremstyle{plain} % Define theorem styles here based on the plain style (used for theorems, lemmas, propositions)
\newtheorem{theorem}{Theorem}
\newtheorem{lemma}{Lemma}
\newtheorem{prop}{Property}
\title{Lemma 19 without auxillary tree}
\begin{document}
\maketitle
\begin{lemma}[Lemma 19, J.P.]
Let $G$ be a simple graph with average degree $d$; $l$ is any positive integer. For any edge $e$, let $T_e$ be the subgraph induced by vertices at most $l$ from $e$. Then there exists an edge $e$ such that $\lambda_1(T_e) \geq \frac{2l}{l+1}\sqrt{d-1}$.
\end{lemma}
\begin{proof}
Assume $G$ has no degree 1 vertices. For this prove we split each edge into two directed edges of opposite direction; thus there are $dn$ directed edges in $G$. First fix an edge $e = {u_0v_0}$. We recursively define the non-backtracking distribution:
$$p_0(e) = 1, p_0(e') = 0(e'\neq e)$$
and for any edge $e = uv$ and integer $1\leq k\leq l$:
$$p_k(uv) = \sum_{w\sim u, w\neq v} \frac{p_{k - 1}(wu)}{d(u) - 1}$$
Finally, we consider the Rayleigh quotient of the vector $h = (P(v))_{v\in V}$, where
$$P(v) = \sqrt{\sum_{k = 0}^l\sum_{u\sim v} p_k(uv)}$$
As $p_k$ is a probability distribution for all $k$,
$$h^th = l+1$$
We now estimate $P(u)P(v)$ for all $u\sim v$. We have
$$P(u)P(v) = \sqrt{ \sum_{k = 0}^l\sum_{w\sim u} p_k(wu) \sum_{k = 0}^l\sum_{w'\sim v} p_k(w'v)}$$
splitting each summation into two parts,
$$P(u)P(v) = \sqrt{ (\sum_{k = 0}^l\sum_{w\sim u, w\neq v} p_k(wu) + \sum_{k = 0}^lp_k(vu))(\sum_{k = 0}^lp_k(uv) + \sum_{k = 0}^l\sum_{w'\sim v, w'\neq u} p_k(w'v) })$$
Now we use Cauchy-Schwarz inequality as follows. We match $p_k(vu)$ with $\sum_{w'\sim v, w'\neq u} p_{k-1}(w'v)$ for $1\leq k\leq l$, match $p_{k}(uv)$ with $\sum_{w\sim u, w\neq v} p_{k-1}(wu)$ for $1\leq k\leq l$, and discard the rest of the terms. Thus
$$P(u)P(v) \geq \sum_{k = 1}^l \sqrt{p_{k}(uv)\sum_{w\sim u, w\neq v} p_{k-1}(wu)} + \sum_{k = 1}^l\sqrt{p_k(vu)\sum_{w'\sim v, w'\neq u} p_{k-1}(w'v)}$$
Using the definition of $p_k(vu)$, then term-by-term Cauchy-Schwarz,
$$p_{k}(uv)\sum_{w\sim u, w\neq v} p_{k-1}(wu) = \sum_{w\sim u, w\neq v} \frac{p_{k - 1}(wu)}{d(u) - 1}\sum_{w\sim u, w\neq v} p_{k-1}(wu) \geq (\sum_{w\sim u, w\neq v} \frac{p_{k - 1}(wu)}{\sqrt{d(u) - 1}})^2$$
The same thing can be done for the other term. Thus,
$$P(u)P(v) \geq \sum_{k = 1}^l\sum_{w\sim u, w\neq v} \frac{p_{k - 1}(wu)}{\sqrt{d(u) - 1}} + \sum_{k = 1}^l\sum_{w'\sim v, w'\neq u} \frac{p_{k - 1}(w'v)}{\sqrt{d(v) - 1}}$$
Now, by symmetry,
$$h^tAh = \sum_{u\sim v} P(u)P(v) \geq 2\sum_{u\sim v}\sum_{k = 1}^l\sum_{w\sim u, w\neq v} \frac{p_{k - 1}(wu)}{\sqrt{d(u) - 1}}$$
Note that for each fixed $w,u$, the term $\frac{p_{k - 1}(wu)}{\sqrt{d(u) - 1}}$ is in the sum for all $v\sim u$, $v\neq w$. There is exactly $d(u) - 1$ of them. Thus,
$$h^tAh \geq 2\sum_{k = 0}^{l-1}\sum_{w\sim u} p_k(wu)\sqrt{d(u) - 1}$$
By definition,
$$\lambda_1(T_e) \geq \frac{h^tAh}{h^th}\geq \frac{2\sum_{k = 0}^{l-1}\sum_{w\sim u} p_k(wu)\sqrt{d(u) - 1}}{l+1}$$
Finally we take expectation over a uniform distribution over all possible $e$,
$$\mathbb{E}(\lambda_1(T_e)) \geq \frac{2}{l+1}\sum_{k = 0}^{l-1}\sum_{w\sim u} \mathbb{E}(p_k(wu))\sqrt{d(u) - 1} = \frac{2l}{(l+1)dn}\sum_u d(u)\sqrt{d(u) - 1}$$
as $\mathbb{E}(p_k(wu)) = \frac{1}{dn}$ for all $k$ and $wu$.
\end{proof}
\end{document}
